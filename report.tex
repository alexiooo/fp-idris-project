\documentclass[12pt,a4paper]{article}
\usepackage{etex,datetime,setspace,latexsym,amssymb,amsmath,amsthm}
\usepackage{fancybox,dialogue,float,wrapfig,enumerate,microtype}
\usepackage{verbatim,xcolor,multicol,titlesec,tabularx,mdframed}

\usepackage[utf8]{inputenc}
\usepackage[pdftex]{hyperref}
\usepackage[margin=2cm,bottom=3cm,footskip=15mm]{geometry}
\parindent0cm
\parskip1em

\usepackage{tikz}
\usetikzlibrary{arrows,trees,positioning,shapes,patterns}
\usetikzlibrary{intersections,calc,fpu,decorations.pathreplacing}

\usepackage[T1]{fontenc} % better fonts

\usepackage{comment}
\excludecomment{hidden}

% Haskell code listings in our own style
\usepackage{listings,color}
\definecolor{lightgrey}{gray}{0.35}
\definecolor{darkgrey}{gray}{0.20}
\definecolor{lightestyellow}{rgb}{1,1,0.92}
\definecolor{dkgreen}{rgb}{0,.2,0}
\definecolor{dkblue}{rgb}{0,0,.2}
\definecolor{dkyellow}{cmyk}{0,0,.7,.5}
\definecolor{lightgrey}{gray}{0.4}
\definecolor{gray}{gray}{0.50}
\lstset{
  language        = Haskell,
  basicstyle      = \scriptsize\ttfamily,
  keywordstyle    = \color{dkblue},     stringstyle     = \color{red},
  identifierstyle = \color{dkgreen},    commentstyle    = \color{gray},
  showspaces      = false,              showstringspaces= false,
  rulecolor       = \color{gray},       showtabs        = false,
  tabsize         = 8,                  breaklines      = true,
  xleftmargin     = 8pt,                xrightmargin    = 8pt,
  frame           = single,             stepnumber      = 1,
  aboveskip       = 2pt plus 1pt,
  belowskip       = 8pt plus 3pt
}
\lstnewenvironment{code}[0]{}{}
\lstnewenvironment{showCode}[0]{\lstset{numbers=none}}{} % only shown, not compiled

% will the real phi please stand up
\renewcommand{\phi}{\varphi}

% load hyperref as late as possible for compatibility
\usepackage[pdftex]{hyperref}
\hypersetup{pdfborder = {0 0 0}, breaklinks = true}

\newcommand{\mbf}{\mathbf}
\newcommand{\sep}{\ \mbf{|}\ }

\title{My Report}
\author{Me}
\date{\today}
\hypersetup{pdfauthor={Me}, pdftitle={My Report}}

\begin{document}

\maketitle

\begin{abstract}
We give a toy example of a report in \emph{literate programming} style.
The main advantage of this is that source code and documentation can
be written and presented next to each other.
We use the listings package to typeset Haskell source code nicely.
\end{abstract}

\vfill

\tableofcontents

\clearpage

% We include one file for each section. The ones containing code should
% be called something.lhs and also mentioned in the .cabal file.


\section{The Language PCF}\label{sec:pcf}

PCF stands for ``programming with computable functions.'' It is based on LCF, the ``logic of computable functions,'' which was introduced by Dana Scott in \cite{LCF} as a restricted formalism for investigating computation, one better suited to this task than set theory. Gordon Plotkin then introduced PCF in \cite{PCF} in order to investigate the semantics of programming languages. He named this language PCF because it was inspired by the syntax of LCF. In what follows, we give a quick introduction to the language PCF. The reader may consult \cite{LambdaNotes} for a more detailed introduction.

\subsection{Types and Terms}

PCF is a programming language that is very similar to the simply-typed lambda calculus. Instead of type variables, it has concrete types $\mbf{bool}$ and $\mbf{nat}$. It also has the classical type constructors $\to$, $\times$ and $1$. Formally, types are defined by the following BNF:

\[ A, B := \mbf{bool} \sep \mbf{nat} \sep A\to B \sep A\times B \sep 1. \]

The terms for PCF are those of the simply-typed lambda calculus, plus some concrete terms. In BNF:

\begin{align*}
    M, N, P := x &\sep MN \sep \lambda x^A.M \sep \langle M, N\rangle \sep \pi_1M \sep \pi_2M \sep * \\
    &\sep \mbf{T} \sep \mbf{F} \sep \mbf{zero} \sep \mbf{succ}\,(M) \sep \mbf{pred}\,(M) \\
    &\sep \mbf{iszero}\,(M) \sep \mbf{if}\,M\,\mbf{then}\,N\,\mbf{else}\,P \sep \mbf{Y}(M).
\end{align*}

The $\mbf{Y}$ constructor returns a fixpoint of the given term. It is what makes the language Turing complete. Without adding $\Y$ to PCF, we would be able to prove that every well-typed term eventually reduces to a term that cannot be reduced further. That is, without $\Y$, every program written in PCF halts, but there are computable functions that do not halt on some inputs, so PCF without $\Y$ cannot be Turing complete. An example of how $\Y$ produces non-halting PCF programs is given by $\Y(\lambda x^{\textbf{nat}}. \textbf{succ } x),$ whose reduction never halts:

% Reducing (lambda x . succ x) to succ makes this easier to read, but is succ a term of PCF?

$$\Y(\lambda x^{\textbf{nat}}. \textbf{succ } x) \to^* \textbf{succ}(\Y(\lambda x^{\textbf{nat}}. \textbf{succ } x)) \to^* \succ(\succ(\Y(\lambda x^{\textbf{nat}}. \textbf{succ } x))) \to^* ... .$$

One may think of $\Y$ as a way to introduce recursion into PCF. For example, the addition function is most naturally defined by recursion, and it can be defined using $\Y$ as follows:

\[ + := \Y(\lambda f^{\mbf{nat}\to\mbf{nat}}m^\mbf{nat}n^\mbf{nat}.\mbf{if}\,\mbf{iszero}(m)\,\mbf{then}\,n\,\mbf{else }\,\mbf{succ}(f \mbf{pred}(m) n)). \]

\subsection{Typing Rules}

The reader may notice that among the well-formed terms, some do not make any sense. Such terms are, for example, $\mbf{iszero}(\mbf{T})$ or $\pi_1(\lambda x^\mbf{nat}.x)$. We should thus specify a type for each term and specify which constructions can be applied to which types. We start by giving the typing rules for the fragment of PCF that is the simply-typed lambda calculus with finite products:

\begin{multicols}{2}
    
    \begin{prooftree} % unit
    \AxiomC{}
    \UnaryInfC{$\Gamma \proves * : 1$}
    \end{prooftree}
    
    \begin{prooftree} % var
    \AxiomC{}
    \UnaryInfC{$\Gamma, x:A \proves x:A$}
    \end{prooftree}
    
    \begin{prooftree} % abs
    \AxiomC{$\Gamma, x:A\proves M:B$}
    \UnaryInfC{$\Gamma\proves\lambda x^A.M : A\to B$}
    \end{prooftree}
    
    \begin{prooftree} % app
    \AxiomC{$\Gamma\proves M:A\to B\qquad \Gamma\proves N:A$}
    \UnaryInfC{$\Gamma\proves MN:B$}
    \end{prooftree}
    
    \begin{prooftree} % pair
    \AxiomC{$\Gamma\proves M:A\qquad\Gamma\proves N:B$}
    \UnaryInfC{$\Gamma\proves \langle M,N\rangle:A\times B$}
    \end{prooftree}
    
    \begin{prooftree} % pi_1
    \AxiomC{$\Gamma\proves M:A\times B$}
    \UnaryInfC{$\Gamma\proves\pi_1M:A$}
    \end{prooftree}
    
    
    
    \begin{prooftree} % pi_2
    \AxiomC{$\Gamma\proves M:A\times B$}
    \UnaryInfC{$\Gamma\proves\pi_2M:B.$}
    \end{prooftree}
    
    
\end{multicols}

We now give the typing rules for the rest of PCF:

\begin{multicols}{2}
    \begin{prooftree} % T
    \AxiomC{}
    \UnaryInfC{$\Gamma \proves \mathbf{T} : \mathbf{bool}$}
    \end{prooftree}
    
    \begin{prooftree} % zero
    \AxiomC{}
    \UnaryInfC{$\Gamma \proves \mathbf{zero} : \mathbf{nat}$}
    \end{prooftree}
    
    \begin{prooftree} % pred
    \AxiomC{$\Gamma \proves M : \mathbf{nat}$}
    \UnaryInfC{$\Gamma \proves \mathbf{pred}(M) : \mathbf{nat}$}
    \end{prooftree}
    
    \begin{prooftree} % if-then-else
    \AxiomC{$\Gamma \proves M : \mathbf{bool}\qquad\Gamma \proves N : A\qquad\Gamma \proves P : A$}
    \UnaryInfC{$\Gamma \proves \textbf{if } M \textbf{ then } N \textbf{ else } P : A$}
    \end{prooftree}

    \begin{prooftree} % F
    \AxiomC{}
    \UnaryInfC{$\Gamma \proves \mathbf{F} : \mathbf{bool}$}
    \end{prooftree}
    
    \begin{prooftree} % succ
    \AxiomC{$\Gamma \proves M : \mathbf{nat}$}
    \UnaryInfC{$\Gamma \proves \mathbf{succ}(M) : \mathbf{nat}$}
    \end{prooftree}
    
    \begin{prooftree} % iszero
    \AxiomC{$\Gamma \proves M : \mathbf{nat}$}
    \UnaryInfC{$\Gamma \proves \mathbf{iszero}(M) : \mathbf{bool}$}
    \end{prooftree}
    
    \begin{prooftree} % Y
    \AxiomC{$\Gamma \proves M : A \to A$}
    \UnaryInfC{$\Gamma \proves \mathbf{Y}(M) : A.$}
    \end{prooftree}
\end{multicols}

\subsection{Reduction}

PCF is a programming language in which PCF-terms are programs. How do we run such a program? We reduce it to its simplest form. Note that the syntax of a PCF-term is not sufficient to determine its reduction behavior. For example, we have two sensible ways to reduce the term $\pred(\succ((\lambda x^{\textbf{nat}}.x)\zero)),$ as shown below: 

\[
\begin{tikzcd}[ampersand replacement=\&]
\& \mathbf{pred} (\mathbf{succ} ((\lambda x^{\textbf{nat}}.x)\mathbf{zero})) \ar[ld] \ar[rd]\& \\
(\lambda x^{\textbf{nat}}.x)\textbf{zero} \ar[rd] \& \& \textbf{pred} ( \textbf{succ} (\mathbf{zero})) \ar[ld] \\
\& \textbf{zero} \& \\
\end{tikzcd}
\]

In order to view PCF-terms as programs, we need to specify a reduction order for PCF-terms so that they can be run deterministically. This reduction order is what makes PCF a programming language, rather than just a calculus like the simply-typed lambda calculus. Below, we specify this reduction order in in the small-step reduction style, while noting that it is also possible to do this in a big-step reduction style.

%%%%% small-step for simply-typed lambda calculus

Small-step reduction specifies how to reduce a PCF-term one step at a time. For example, the reduction $\pred(\succ((\lambda x^{\textbf{nat}}.x)\zero)) \to \pred(\succ(\zero))$ is one small-step reduction that the rules below specify. If we reduce once more, $\pred(\succ(\zero)) \to \zero$, we arrive at the \textit{value} $\zero$, which we think of as the result of running the program $\pred(\succ((\lambda x^{\textbf{nat}}.x)\zero)).$ In general, the form of values is given by the following BNF, where $M$ and $N$ are any PCF-terms:
$$ V := \T \sep \F \sep \zero \sep \succ(V) \sep * \sep \langle M,N \rangle \sep \lambda x^A.M.$$

We begin by giving the small-step reduction rules for the fragment of PCF that is the simply-typed lambda calculus with finite products:

\begin{multicols}{2}

% small-step unit
\begin{prooftree}
\AxiomC{$M : 1$}
\AxiomC{$M \neq *$}
\BinaryInfC{$M \to *$}
\end{prooftree}

% small-step functions
\begin{prooftree}
\AxiomC{}
\UnaryInfC{$(\lambda x^A . M)N \to M[N/x]$}
\end{prooftree}

\begin{prooftree}
\AxiomC{$M \to N$}
\UnaryInfC{$MP \to NP$}
\end{prooftree}

%small-step products
\begin{prooftree}
\AxiomC{}
\UnaryInfC{$\pi_1 \langle M_1,M_2 \rangle \to M_1$}
\end{prooftree}

\begin{prooftree}
\AxiomC{}
\UnaryInfC{$\pi_2 \langle M_1, M_2 \rangle \to M_2$}
\end{prooftree}

\begin{prooftree}
\AxiomC{$M \to N$}
\UnaryInfC{$\pi_i M \to \pi_i N.$}
\end{prooftree}
\end{multicols}

We now give the small-step reduction rules for the rest of PCF:

\begin{multicols}{2}
%%%%% small-step for PCF

%small-step Y
\begin{prooftree}
\AxiomC{}
\UnaryInfC{$\Y(M) \to M(\Y(M))$}
\end{prooftree}

%small-step succ
\begin{prooftree}
\AxiomC{$M \to N$}
\UnaryInfC{$\succ(M) \to \succ(N)$}
\end{prooftree}




%small-step pred
\begin{prooftree}
\AxiomC{}
\UnaryInfC{$\pred (\zero) \to \zero$}
\end{prooftree}

\begin{prooftree}
\AxiomC{$M \to N$}
\UnaryInfC{$\pred(M) \to \pred(N)$}
\end{prooftree}

\begin{prooftree}
\AxiomC{}
\UnaryInfC{$\pred(\succ(V)) \to V$}
\end{prooftree}

%small-step iszero
\begin{prooftree}
\AxiomC{}
\UnaryInfC{$\iszero (\zero) \to \T$}
\end{prooftree}

\begin{prooftree}
\AxiomC{$M \to N$}
\UnaryInfC{$\iszero(M) \to \iszero(N)$}
\end{prooftree}

\begin{prooftree}
\AxiomC{}
\UnaryInfC{$\iszero(\succ(V)) \to \F$}
\end{prooftree}



% small-step if then else
\begin{prooftree}
\AxiomC{}
\UnaryInfC{$\textbf{if } \T \textbf{ then } N \textbf{ else } P \to N$}
\end{prooftree}

\begin{prooftree}
\AxiomC{}
\UnaryInfC{$\textbf{if } \F \textbf{ then } N \textbf{ else } P \to P$}
\end{prooftree}

\end{multicols}

\begin{prooftree}
\AxiomC{$M \to M'$}
\UnaryInfC{$\textbf{if } M \textbf{ then } N \textbf{ else } P \to \textbf{if } M' \textbf{ then } N \textbf{ else }P.$}
\end{prooftree}



Small-steps can be concatenated. For example, the small-steps $\pred(\succ((\lambda x^{\textbf{nat}}.x)\zero)) \to \pred(\succ(\zero)) \to \zero$ can be concatenated to $\pred(\succ((\lambda x^{\textbf{nat}}.x)\zero)) \to^* \zero,$ where $\to^*$ denotes finitely many small-step reductions. Since $\zero$ is a value, we say that $\pred(\succ((\lambda x^{\textbf{nat}}.x)\zero))$ \textit{evaluates} to $\zero.$ In general, if $M \to^* V$, where $V$ is a value, we say that $M$ evaluates to $V.$

The notion of evaluating to a value is what gives us the big-step style of specifying reduction order. We can give rules that directly axiomatize big-step reduction, usually denoted by $\Downarrow,$ with the intention that $\Downarrow$ coincides with evaluating to a value via $\to^*.$ We will not give the rules for $\Downarrow,$ however, because our implementation of big-step reduction in Idris closely follows the idea that a big-step is composed of multiple small-steps that result in a value.


\begin{comment}
%%%%% big-step for simply-typed lambda calculus

\begin{multicols}{2}

%big-step unit
\begin{prooftree}
\AxiomC{$M : 1$}
\UnaryInfC{$M \Downarrow *$}
\end{prooftree}

%big-step function
\begin{prooftree}
\AxiomC{}
\UnaryInfC{$\lambda x^A.M \Downarrow \lambda x^A .M$}
\end{prooftree}

\begin{prooftree}
\AxiomC{$M \Downarrow \lambda x^A. M'$}
\AxiomC{$M'[N/x]\Downarrow V$}
\BinaryInfC{$MN \Downarrow V$}
\end{prooftree}

\columnbreak
% big-step product
\begin{prooftree}
\AxiomC{}
\UnaryInfC{$\langle M,N \rangle \Downarrow \langle M,N \rangle$}
\end{prooftree}

\begin{prooftree}
\AxiomC{$M \Downarrow \langle M_1,M_2 \rangle$}
\AxiomC{$M_1 \Downarrow V$}
\BinaryInfC{$\pi_1 M \Downarrow V$}
\end{prooftree}



\begin{prooftree}
\AxiomC{$M \Downarrow \langle M_1,M_2 \rangle $}
\AxiomC{$M_2 \Downarrow V$}
\BinaryInfC{$\pi_2 M \Downarrow V$}
\end{prooftree}
\end{multicols}

We now give the big-step reduction rules for the rest of PCF:

%%%%% big-step for PCF 

\begin{multicols}{2}

% big-step Y
\begin{prooftree}
\AxiomC{$M(\mathbf{Y}(M)) \Downarrow V$}
\UnaryInfC{$\mathbf{Y}(M) \Downarrow V$}
\end{prooftree}

% big-step boolean
\begin{prooftree}
\AxiomC{}
\UnaryInfC{$\mathbf{T} \Downarrow \mathbf{T}$}
\end{prooftree}

\begin{prooftree}
\AxiomC{}
\UnaryInfC{$\textbf{F} \Downarrow \textbf{F}$} 
\end{prooftree}

% big-step nat succ
\begin{prooftree}
\AxiomC{\text{}}
\UnaryInfC{$\textbf{zero} \Downarrow \mathbf{zero}$}
\end{prooftree}

\begin{prooftree}
\AxiomC{$M \Downarrow V$}
\UnaryInfC{$\succ (M) \Downarrow \succ (V)$}
\end{prooftree}

% big-step nat pred
\begin{prooftree}
\AxiomC{$M \Downarrow \mathbf{zero}$}
\UnaryInfC{$\mathbf{pred}(M) \Downarrow \mathbf{zero}$}
\end{prooftree}

\begin{prooftree}
\AxiomC{$M \Downarrow \mathbf{succ}(V)$}
\UnaryInfC{$\mathbf{pred}(M) \Downarrow V$}
\end{prooftree}

% big-step iszero
\begin{prooftree}
\AxiomC{$M \Downarrow \mathbf{zero}$}
\UnaryInfC{$\mathbf{iszero}(M) \Downarrow \mathbf{T}$}
\end{prooftree}

\begin{prooftree}
\AxiomC{$M \Downarrow \mathbf{succ}(V)$}
\UnaryInfC{$\mathbf{iszero}(M) \Downarrow \mathbf{F}$}
\end{prooftree}

% big-step if then else
\begin{prooftree}
\AxiomC{$M \Downarrow \T$}
\AxiomC{$N \Downarrow V$}
\BinaryInfC{$\textbf{if } M \textbf{ then } N \textbf{ else } P \Downarrow V$}
\end{prooftree}

\begin{prooftree}
\AxiomC{$M \Downarrow \F$}
\AxiomC{$P \Downarrow V$}
\BinaryInfC{$\textbf{if } M \textbf{ then } N \textbf{ else } P \Downarrow V$}
\end{prooftree}

\end{multicols}

\end{comment}

\subsection{Implementing Terms}

\begin{hidden}
module Lib.Terms

import public Lib.Types

import public Data.Fin  -- needed publically, since we publically export types that reference Fin
import public Data.Vect

%default total
\end{hidden}

We now want to define terms. We use de Bruijn indices to represent bound variables.
This is an elegant way to deal with alpha-equivalence.

\begin{code}
public export
Var : Nat -> Type
Var = Fin
\end{code}

We also keep track of (an upper bound on) free variables in the type:
\lstinline{PCFTerm} n encodes terms with at most n free variables.

\begin{code}
public export
data PCFTerm : Nat -> Type where
  V          : Var k -> PCFTerm k                                -- variables
  L          : PCFType -> PCFTerm (S k) -> PCFTerm k             -- lambda
  IfThenElse : PCFTerm k -> PCFTerm k -> PCFTerm k -> PCFTerm k  -- if-then-else construct
  App        : PCFTerm k -> PCFTerm k -> PCFTerm k               -- application
  Pair       : PCFTerm k -> PCFTerm k -> PCFTerm k               -- pairing
  Fst        : PCFTerm k -> PCFTerm k                            -- first projection
  Snd        : PCFTerm k -> PCFTerm k                            -- second projection
  Succ       : PCFTerm k -> PCFTerm k                            -- successor
  Pred       : PCFTerm k -> PCFTerm k                            -- predecessor
  IsZero     : PCFTerm k -> PCFTerm k                            -- iszero predicate
  Y          : PCFTerm k -> PCFTerm k                            -- fixpoint / Y-combinator
  T          : PCFTerm k                                         -- true
  F          : PCFTerm k                                         -- false
  Zero       : PCFTerm k                                         -- zero
  Unit       : PCFTerm k                                         -- unit value (*)
\end{code}

Of special interest are the closed terms, those without any free variables

\begin{code}
public export
ClosedPCFTerm : Type
ClosedPCFTerm = PCFTerm 0
\end{code}

Remember that the type only gives an upper bound, so an inhabitant of say
\lstinline{PCFType 3} might still be closed.

The following will try to strengthen any such term. This really is just a
wrapper around Fin.strengthen, with straightforward recursive cases,
so we detail only variables and lambdas.   %%%%%%%%%%%%%%% Add details

\begin{code}
strengthen : {k : _} -> PCFTerm (S k) -> Maybe (PCFTerm k)
strengthen (V v)              = map V          $ Fin.strengthen v
strengthen (L t m)            = map (L t)      $ strengthen m
strengthen (IfThenElse m n p) = map IfThenElse (strengthen m) <*> strengthen n <*> strengthen p
strengthen (App m n)          = map App        (strengthen m) <*> strengthen n
strengthen (Pair m n)         = map Pair       (strengthen m) <*> strengthen n
strengthen (Fst m)            = map Fst        $ strengthen m
strengthen (Snd m)            = map Snd        $ strengthen m
strengthen (Succ m)           = map Succ       $ strengthen m
strengthen (Pred m)           = map Pred       $ strengthen m
strengthen (IsZero m)         = map IsZero     $ strengthen m
strengthen (Y m)              = map Y          $ strengthen m
strengthen T                  = Just T
strengthen F                  = Just F
strengthen Zero               = Just Zero
strengthen Unit       = Just Unit
\end{code}

\begin{code}
public export
tryClose : {k : _} -> PCFTerm k -> Maybe ClosedPCFTerm
tryClose {k} t = case k of
                 0      => Just t
                 (S l) => strengthen t >>= tryClose
\end{code}

As for types, we want terms to be comparable. The important case is
lambda-abstraction. We are using de Bruijn indices, which make comparing terms
very easy.

\begin{code}
public export partial
implementation Eq (PCFTerm k) where

  V v              == V w              = v == w
  L a m            == L b n            = a == b && m == n
  IfThenElse m n p == IfThenElse q r s = m == q && n == r && p == s
  App m n          == App p q          = m == p && n == q
  Pair m n         == Pair p q         = m == p && n == q
  Fst m            == Fst n            = m == n
  Snd m            == Snd n            = m == n
  Succ m           == Succ n           = m == n
  Pred m           == Pred n           = m == n
  IsZero m         == IsZero n         = m == n
  Y m              == Y n              = m == n
  T                == T                = True
  F                == F                = True
  Zero             == Zero             = True
  Unit             == unit             = True
  _                == _                = False
\end{code}

We can also implement a show function. The implementation in itself is not
interesting, so we omit the details.

\begin{hidden}
public export
Show (PCFTerm k) where
  show (V x)              = show x
  show (L t m)            = "(λ" ++ show t ++ "." ++ show m ++ ")"
  show (IfThenElse m n p) = "if " ++ show m ++ " then " ++ show n ++ " else " ++ show p
  show (App m n)          = "(" ++ show m ++ show n ++ ")"
  show (Pair m n)         = "<" ++ show m ++ ", " ++ show n ++ ">"
  show (Fst m)            = "π₁(" ++ show m ++ ")"
  show (Snd m)            = "π₂(" ++ show m ++ ")"
  show (Succ m)           = "succ (" ++ show m ++ ")"
  show (Pred m)           = "pred (" ++ show m ++ ")"
  show (IsZero m)         = "iszero (" ++ show m ++ ")"
  show (Y m)              = "Y(" ++ show m ++ ")"
  show T                  = "T"
  show F                  = "F"
  show Zero               = "zero"
  show Unit               = "*"
\end{hidden}


\addcontentsline{toc}{section}{Bibliography}
\bibliographystyle{alpha}
\bibliography{references.bib}

\end{document}
