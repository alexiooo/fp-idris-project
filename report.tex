\documentclass[12pt,a4paper]{article}
\input{latexmacros.tex}

%packages
\usepackage{bussproofs}
\usepackage{amsmath}
\usepackage{tikz-cd}

%commands
\newcommand{\proves}{\vdash}
\renewcommand{\succ}{\textbf{succ}}
\newcommand{\pred}{\textbf{pred}}
\newcommand{\iszero}{\textbf{iszero}}
\newcommand{\T}{\textbf{T}}
\newcommand{\F}{\textbf{F}}
\newcommand{\Y}{\textbf{Y}}
\newcommand{\zero}{\textbf{zero}}


\title{My Report}
\author{Me}
\date{\today}
\hypersetup{pdfauthor={Me}, pdftitle={My Report}}

\begin{document}

\maketitle

\begin{abstract}
We give a toy example of a report in \emph{literate programming} style.
The main advantage of this is that source code and documentation can
be written and presented next to each other.
We use the listings package to typeset Haskell source code nicely.
\end{abstract}

\vfill

\tableofcontents

\clearpage

% We include one file for each section. The ones containing code should
% be called something.lhs and also mentioned in the .cabal file.


\section{The Language PCF}

This section is intended as a quick introduction to the language PCF. The reader may consult \cite{LambdaNotes} for a more detailed introduction.

\subsection{Types and Terms}

PCF is a programming language that is very similar to the simply typed lambda-calculus. Instead of type variables, it has concrete type $\mbf{bool}$ and $\mbf{nat}$. It also has the classical type constructors $\to$, $\times$ and $1$. Formally, types are defined by the following BNF

\[ A, B := \mbf{bool} \sep \mbf{nat} \sep A\to B \sep A\times B \sep 1. \]

The terms for PCF are those of the simply-typed lambda calculus, plus some concrete terms. In BNF:

\begin{align*}
    M, N, P := x &\sep MN \sep \lambda x^A.M \sep \langle M, N\rangle \sep \pi_1M \sep \pi_2M \sep * \\
    &\sep \mbf{T} \sep \mbf{F} \sep \mbf{zero} \sep \mbf{succ}\,(M) \sep \mbf{pred}\,(M) \\
    &\sep \mbf{iszero}\,(M) \sep \mbf{if}\,M\,\mbf{then}\,N\,\mbf{else}\,P \sep \mbf{Y}(M).
\end{align*}

The $\mbf{Y}$ constructor returns a fixpoint of the given term. It is what makes the language Turing complete. For example, the addition function can be defined as

\[ + := Y(\lambda f^{\mbf{nat}\to\mbf{nat}}m^\mbf{nat}n^\mbf{nat}.\mbf{if}\,\mbf{iszero}(m)\,\mbf{then}\,n\,\mbf{else}\,\mbf{succ}(f \mbf{pred}(m) n)). \]

\subsection{Typing rules}

The reader may notice that among the well-formed terms, some do not make any sense. For example, terms such as $\mbf{iszero}(\mbf{T})$ or $\pi_1(\lambda x^\mbf{nat}.x)$ do not make sense. We should thus specify a type for each term and specify which constructions can be applied to which types.



 \begin{prooftree}
 \AxiomC{}
 \UnaryInfC{$\Gamma, x:A \proves x:A$}
 \end{prooftree}

\begin{prooftree}
\AxiomC{$\Gamma\proves M:A\to B\qquad \Gamma\proves N:A$}
\UnaryInfC{$\Gamma\proves MN:B$}
\end{prooftree}

\begin{prooftree}
\AxiomC{$\Gamma\proves M:A\times B$}
\UnaryInfC{$\Gamma\proves\pi_1M:A$}
\end{prooftree}


\begin{prooftree}
\AxiomC{}
\UnaryInfC{$\Gamma \proves * : 1$}
\end{prooftree}

\begin{prooftree}
\AxiomC{$\Gamma, x:A\proves M:B$}
\UnaryInfC{$\Gamma\proves\lambda x^A.M : A\to B$}
\end{prooftree}

\begin{prooftree}
\AxiomC{$\Gamma\proves M:A\times B$}
\UnaryInfC{$\Gamma\proves\pi_2M:B$}
\end{prooftree}



\begin{prooftree}
\AxiomC{$\Gamma\proves M:A \qquad \Gamma\proves N:B$}
\UnaryInfC{$\Gamma\proves \langle M,N\rangle:A\times B$}
\end{prooftree}

%% include PCF typing rules

\begin{prooftree}
\AxiomC{}
\UnaryInfC{$\Gamma \proves \mathbf{T} : \mathbf{bool}$}
\end{prooftree}  

\begin{prooftree}
\AxiomC{}
\UnaryInfC{$\Gamma \proves \mathbf{F} : \mathbf{bool}$}
\end{prooftree}

\begin{prooftree}
\AxiomC{$\Gamma \proves M : A \to A$}
\UnaryInfC{$\Gamma \proves \mathbf{Y}(M) : A$}
\end{prooftree}

\begin{prooftree}
\AxiomC{}
\UnaryInfC{$\Gamma \proves \mathbf{zero} : \mathbf{nat}$}
\end{prooftree}

\begin{prooftree}
\AxiomC{$\Gamma \proves M : \mathbf{nat}$}
\UnaryInfC{$\Gamma \proves \mathbf{succ}(M) : \mathbf{nat}$}
\end{prooftree}

\begin{prooftree}
\AxiomC{$\Gamma \proves M : \mathbf{nat}$}
\UnaryInfC{$\Gamma \proves \mathbf{pred}(M) : \mathbf{nat}$}
\end{prooftree}

\subsection{Reduction}

TBA.

\subsubsection{Small-step Reduction}

TBA.

\subsubsection{Big-step Reduction}

TBA.

\section{A Prime Implementation of PCF}

\section{Implementing Terms}

\begin{hidden}
module Lib.Terms

import public Lib.Types

import public Data.Fin  -- needed publically, since we publically export types that reference Fin
import public Data.Vect

%default total
\end{hidden}

PCF is a simple language that models computing. Its types are as follows.

%%% Include Lib.Types here

We begin by defining terms. We use de Bruijn indices to representent bound
variables. This is an elegant way to deel with alpha-equivalence.
We also keep track of (an upper bound on) free variables in the type;
PCFTerm n encodes terms with at most n free variables

\begin{code}
||| Var k is a De-Bruijn index less than k
public export
Var : Nat -> Type
Var = Fin

namespace Symbol
  public export
  data Symbol : (0 ar : Nat) -> Type where
    IfElse : Symbol 3       -- if-then-else construct
    App    : Symbol 2       -- application
    Pair   : Symbol 2       -- pairing
    Fst    : Symbol 1       -- first projection
    Snd    : Symbol 1       -- second projection
    Succ   : Symbol 1       -- successor
    Pred   : Symbol 1       -- predecessor
    IsZero : Symbol 1       -- is zero predicate
    Y      : Symbol 1       -- fixpoint / Y-combinator
    T      : Symbol 0       -- true
    F      : Symbol 0       -- false
    Zero   : Symbol 0       -- zero value
    Unit   : Symbol 0       -- unit value (*)
\end{code}

\begin{code}
public export
data PCFTerm : Nat -> Type where
  V    : Var k -> PCFTerm k                             -- variables
  L    : PCFType   -> PCFTerm (S k) -> PCFTerm k        -- lambda
  S    : Symbol ar -> Vect ar (PCFTerm k) -> PCFTerm k  -- other symbols
\end{code}

Of special interest are the closed terms, those without any free variables

\begin{code}
public export
ClosedPCFTerm : Type
ClosedPCFTerm = PCFTerm 0
\end{code}

\input{Lib/Examples/SumExample}

Remember that the type only gives an upper bound, so an inhabitant of say PCFType 3 might still
be closed. The following will try to strengthen any such term.
This really is just a wrapper around Fin.strengthen, with straightforward recursive cases, so we
detail only variables and lambdas.

\begin{code}
strengthen : {k :_} -> PCFTerm (S k) -> Maybe (PCFTerm k)
strengthenVect : {k :_} -> Vect n (PCFTerm (S k)) -> Maybe (Vect n (PCFTerm k))
strengthenVect (y::ys) = [| (strengthen y) :: (strengthenVect ys) |]
strengthenVect [] = Just []
\end{code}

\begin{code}
strengthen (V v)    = Fin.strengthen v >>= Just . V
strengthen (L t m)  = strengthen m     >>= Just . L t
strengthen (S s ar) = Just (S s !(strengthenVect ar))
\end{code}

\begin{code}
public export
tryClose : {k:_} -> PCFTerm k -> Maybe ClosedPCFTerm
tryClose {k} t = case k of
                 0      => Just t
                 (S k') => strengthen t >>= tryClose
\end{code}

Sadly, Idris does not have an equivalent of Haskell's `deriving` statement, so we'll have to
implement equality ourselves. We omit the details here and in any other similarly trivial
implementation blocks

\begin{hidden}
implementation Eq (Symbol k) where
  IfElse == IfElse = True
  App    == App    = True
  Pair   == Pair   = True
  Fst    == Fst    = True
  Snd    == Snd    = True
  Succ   == Succ   = True
  Pred   == Pred   = True
  IsZero == IsZero = True
  Y      == Y      = True
  T      == T      = True
  F      == F      = True
  Zero   == Zero   = True
  Unit   == Unit   = True
  _      == _      = False
\end{hidden}

Eq requires that it's arguments are of the same type, so it only works for symbols of known arity.
s1 ~~ s2 holds iff s1 == s2, but the former will typecheck even if the arities don't match.

\begin{code}
namespace Symbol
  infixr 6 ~~
  public export
  (~~) : Symbol k -> Symbol l -> Bool
  -- trivial implementation omitted
  IfElse ~~ IfElse = True
  App    ~~ App    = True
  Pair   ~~ Pair   = True
  Fst    ~~ Fst    = True
  Snd    ~~ Snd    = True
  Succ   ~~ Succ   = True
  Pred   ~~ Pred   = True
  IsZero ~~ IsZero = True
  Y      ~~ Y      = True
  T      ~~ T      = True
  F      ~~ F      = True
  Zero   ~~ Zero   = True
  Unit   ~~ Unit   = True
  _      ~~ _      = False
\end{code}

We are now able to define equality for terms. The important case is
lambda-abstraction. We are using de Bruijn indices, which make comparing terms
very easy.

\begin{code}
public export partial
implementation Eq (PCFTerm k) where

  V v         == V w          = v == w
  L a m       == L b n        = a == b && m == n
  S s [a]     == S p [x]      = s == p && a == x
  S s [a,b]   == S p [x,y]    = s == p && a == x && b == y
  S s [a,b,c] == S p [x,y,z]  = s == p && a == x && b == y && c == z
  _           == _            = False
\end{code}

We can also implement a show function. The implementation in itself is not
interesting, so we omit the details.

\begin{hidden}
public export
Show (PCFTerm k) where
  show (V x)   = show x
  show (L x y) = "( λ " ++ show x ++ " . " ++ show y ++ ")"
  show (S IfElse [p, m, n]) = ?hole_4
  show (S App    [m,n]) = (show m) ++ (show n)
  show (S Pair   [m,n]) = "<" ++ show m ++ ", " ++ show n ++ ">"
  show (S Fst    [m])   = "π₁(" ++ show m ++ ")"
  show (S Snd    [m])   = "π₂(" ++ show m ++ ")"
  show (S Succ   [m])   = "Succ (" ++ show m ++ ")"
  show (S Pred   [m])   = "Pred (" ++ show m ++ ")"
  show (S IsZero [m])   = "IsZero (" ++ show m ++ ")"
  show (S Y      [m])   = "Y (" ++ show m ++ ")"
  show (S T      _)     = "T"
  show (S F      _)     = "F"
  show (S Zero   _)     = "Zero"
  show (S Unit   _)     = "Unit"
\end{hidden}


\addcontentsline{toc}{section}{Bibliography}
\bibliographystyle{alpha}
\bibliography{references.bib}

\end{document}
